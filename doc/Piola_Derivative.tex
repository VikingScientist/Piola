\documentclass[final,1p,times]{elsarticle}

% Packages
\usepackage[hypertexnames=false]{hyperref}
\usepackage{amsmath,amssymb,amsfonts}
\usepackage{amsthm,mathrsfs}
\usepackage{bbm,bm,mathtools,esint}
\usepackage{exscale,relsize}
\usepackage{amscd}
\usepackage{scalerel}
\usepackage{resizegather}
\usepackage{subfig}
\usepackage{tabularx}
\usepackage{tabulary}
\usepackage{rotating}
\usepackage{etoolbox}
\usepackage{empheq}
\usepackage{xcolor}
\usepackage{algorithm}
\usepackage[noend]{algpseudocode}
\usepackage{tikz}
\allowdisplaybreaks

% User-defined commands
\newcommand{\E}{|\hspace{-0.2mm}|\hspace{-0.2mm}|}
\newcommand{\leftE}{\left|\hspace{-0.2mm}\left|\hspace{-0.2mm}\left|}
\newcommand{\rightE}{\right|\hspace{-0.2mm}\right|\hspace{-0.2mm}\right|}
\DeclareMathOperator{\diag}{diag}
%
\newcommand{\bLambda}{\boldsymbol{\Lambda}}
\newcommand{\bmu}{\boldsymbol{\mu}}
\newcommand{\bPsi}{\boldsymbol{\Psi}}
\newcommand{\bpi}{\boldsymbol{\pi}}
\newcommand{\brho}{\boldsymbol{\rho}}
\newcommand{\bsigma}{\boldsymbol{\sigma}}
\newcommand{\bSigma}{\boldsymbol{\Sigma}}
\newcommand{\bzeta}{\boldsymbol{\zeta}}
\newcommand{\bvphi}{\boldsymbol{\varphi}}
\newcommand{\wbM}{\widehat{\bold{M}}}
\newcommand{\wbK}{\widehat{\bold{K}}}
\newcommand{\wbC}{\widehat{\bold{C}}}
\newcommand{\wbA}{\widetilde{\bold{A}}}
\newcommand{\wbf}{\widetilde{\bold{f}}}
\newcommand{\wbu}{\widetilde{\bold{u}}}
%
\DeclareMathOperator{\sech}{sech}
\DeclareMathOperator{\csch}{csch}


\begin{document}
\begin{frontmatter}

\title{Derivative of Piola mapping}

\author[1]{Abdullah Abdulhaque\corref{cor1}}	\ead{abdullah.abdulhaque@ntnu.no}
\affiliation[1]{
	organization={Department of Mathematical Sciences, Norwegian University of Science and Technology},
	city={Trondheim},
	country={Norway}}
	
\begin{abstract}
We present the derivatives of the Piola mapping.
\end{abstract}

\clearpage

\end{frontmatter}



\section{First-order derivatives}\noindent
If we have a mapping $(\xi,\eta)\to(x,y)$, then its Jacobian is defined as
\begin{equation}
J = \begin{bmatrix} x_{\xi} & x_{\eta} \\ y_{\xi} & y_{\eta} \end{bmatrix}
\qquad,\qquad |J| = x_{\xi}y_{\eta}-x_{\eta}y_{\xi}
\end{equation}
From the general inversion formula, we get
\begin{equation}
J^{-1} = \begin{bmatrix} \xi_x & \xi_y \\ \eta_x & \eta_y \end{bmatrix} = 
\frac{1}{|J|}\begin{bmatrix} y_{\eta} & -x_{\eta} \\ -y_{\xi} & x_{\xi} \end{bmatrix}
\end{equation}
From the chain rule, we have an universal formula for the total derivative:
\begin{equation}
f_x = f_{\xi}\xi_x + f_{\eta}\eta_x \qquad,\qquad f_y = f_{\xi}\xi_y + f_{\eta}\eta_y
\end{equation}
The Piola mapping is defined as $J/|J|$. Its first-order derivatives become
\begin{subequations}
\begin{align}
&\frac{\partial}{\partial x}\left[\frac{1}{|J|}J\right] = -\frac{|J|_x}{|J|^2}J+\frac{1}{|J|}J_x \\
&\frac{\partial}{\partial y}\left[\frac{1}{|J|}J\right] = -\frac{|J|_y}{|J|^2}J+\frac{1}{|J|}J_y
\end{align}
\end{subequations}
Further computations yield two similar formulas:
\begin{subequations}
\begin{align}
&|J|_x = \xi_x\widetilde{J}_1+\eta_x\widetilde{J}_2 \\ 
&|J|_y = \xi_y\widetilde{J}_1+\eta_y\widetilde{J}_2
\end{align}
\end{subequations}
\clearpage\noindent
In this setting, the auxiliary functions are
\begin{subequations}
\begin{align}
&\widetilde{J}_1 = x_{\xi\xi}y_{\eta}+x_{\xi}y_{\xi\eta}-x_{\xi\eta}y_{\xi}-x_{\eta}y_{\xi\xi} \\
&\widetilde{J}_2 = x_{\xi\eta}y_{\eta}+x_{\xi}y_{\eta\eta}-x_{\eta\eta}y_{\xi}-x_{\eta}y_{\xi\eta}
\end{align}
\end{subequations}
Likewise, we get the following matrix derivatives:
\begin{subequations}
\begin{align}
&J_x = \xi_x\begin{bmatrix} x_{\xi\xi} & x_{\xi\eta} \\ y_{\xi\xi} & y_{\xi\eta} \end{bmatrix} + \eta_x\begin{bmatrix} x_{\xi\eta} & x_{\eta\eta} \\ y_{\xi\eta} & y_{\eta\eta} \end{bmatrix} \\
&J_y = \xi_y\begin{bmatrix} x_{\xi\xi} & x_{\xi\eta} \\ y_{\xi\xi} & y_{\xi\eta} \end{bmatrix} + \eta_y\begin{bmatrix} x_{\xi\eta} & x_{\eta\eta} \\ y_{\xi\eta} & y_{\eta\eta} \end{bmatrix}
\end{align}
\end{subequations}
Imagine now that we have a vector function on the form
\begin{equation*}
\begin{bmatrix} u \\ v \end{bmatrix} = \frac{1}{|J|}J\begin{bmatrix} U \\ V \end{bmatrix}
\end{equation*}
By using the product rule, we obtain
\begin{subequations}
\begin{align}
&\begin{bmatrix} u \\ v \end{bmatrix}_x = \left(-\frac{|J|_x}{|J|^2}J+\frac{1}{|J|}J_x\right)\begin{bmatrix} U \\ V \end{bmatrix} + \frac{1}{|J|}J\begin{bmatrix} U_{\xi}\xi_x+U_{\eta}\eta_x \\ V_{\xi}\xi_x+V_{\eta}\eta_x \end{bmatrix} \\
&\begin{bmatrix} u \\ v \end{bmatrix}_y = \left(-\frac{|J|_y}{|J|^2}J+\frac{1}{|J|}J_y\right)\begin{bmatrix} U \\ V \end{bmatrix} + \frac{1}{|J|}J\begin{bmatrix} U_{\xi}\xi_y+U_{\eta}\eta_y \\ V_{\xi}\xi_y+V_{\eta}\eta_y \end{bmatrix}
\end{align}
\end{subequations}


\section{Second-order derivatives}\noindent
From the double product rule, we have 
\begin{subequations}
\begin{align}
&\left[\frac{1}{|J|}J\right]_{xx} = \left[\frac{1}{|J|}\right]_{xx}J+2\left[\frac{1}{|J|}\right]_x J_x+\frac{1}{|J|}J_{xx} \\
&\left[\frac{1}{|J|}J\right]_{yy} = \left[\frac{1}{|J|}\right]_{yy}J+2\left[\frac{1}{|J|}\right]_y J_y+\frac{1}{|J|}J_{yy}
\end{align}
\end{subequations}
Applying the same rule on $1/|J|$ yields
\begin{subequations}
\begin{align}
&\left[\frac{1}{|J|}\right]_{xx} = -\frac{|J|_{xx}}{|J|^2}+\frac{|J|_x^2}{|J|^3} \\
&\left[\frac{1}{|J|}\right]_{yy} = -\frac{|J|_{yy}}{|J|^2}+\frac{|J|_y^2}{|J|^3}
\end{align}
\end{subequations}
\clearpage\noindent
As we see from these formulas, the real objective is finding $|J|_{xx}$, $|J|_{yy}$, $J_{xx}$ and $J_{yy}$. We start with $J_{xx}$ and use the total derivative rule:
\begin{align*}
J_{xx}	&= \xi_{xx}\begin{bmatrix} x_{\xi\xi} & x_{\xi\eta} \\ y_{\xi\xi} & y_{\xi\eta} \end{bmatrix}_x + \eta_{xx}\begin{bmatrix} x_{\xi\eta} & x_{\eta\eta} \\ y_{\xi\eta} & y_{\eta\eta} \end{bmatrix}_x \\
		&= \xi_{xx}\begin{bmatrix} x_{\xi\xi\xi}\xi_x+x_{\xi\xi\eta}\eta_x & x_{\xi\xi\eta}\xi_x+x_{\xi\eta\eta}\eta_x \\ y_{\xi\xi\xi}\xi_x+y_{\xi\xi\eta}\eta_x & y_{\xi\xi\eta}\xi_x+y_{\xi\eta\eta}\eta_x \end{bmatrix} + \eta_{xx}\begin{bmatrix} x_{\xi\xi\eta}\xi_x+x_{\xi\eta\eta}\eta_x & x_{\xi\eta\eta}\xi_x+x_{\eta\eta\eta}\eta_x \\ y_{\xi\xi\xi}\xi_x+y_{\xi\xi\eta}\eta_x  & y_{\xi\eta\eta}\xi_x+y_{\eta\eta\eta}\eta_x \end{bmatrix}
\end{align*}
Summarizing, we have arrived at
\begin{subequations}
\begin{align}
J_{xx}	&= \xi_{xx}\begin{bmatrix} x_{\xi\xi\xi}\xi_x+x_{\xi\xi\eta}\eta_x & x_{\xi\xi\eta}\xi_x+x_{\xi\eta\eta}\eta_x \\ y_{\xi\xi\xi}\xi_x+y_{\xi\xi\eta}\eta_x & y_{\xi\xi\eta}\xi_x+y_{\xi\eta\eta}\eta_x \end{bmatrix} + \eta_{xx}\begin{bmatrix} x_{\xi\xi\eta}\xi_x+x_{\xi\eta\eta}\eta_x & x_{\xi\eta\eta}\xi_x+x_{\eta\eta\eta}\eta_x \\ y_{\xi\xi\xi}\xi_x+y_{\xi\xi\eta}\eta_x  & y_{\xi\eta\eta}\xi_x+y_{\eta\eta\eta}\eta_x \end{bmatrix} \\
J_{yy}	&= \xi_{yy}\begin{bmatrix} x_{\xi\xi\xi}\xi_y+x_{\xi\xi\eta}\eta_y & x_{\xi\xi\eta}\xi_y+x_{\xi\eta\eta}\eta_y \\ y_{\xi\xi\xi}\xi_y+y_{\xi\xi\eta}\eta_y & y_{\xi\xi\eta}\xi_y+y_{\xi\eta\eta}\eta_y \end{bmatrix} + \eta_{xx}\begin{bmatrix} x_{\xi\xi\eta}\xi_y+x_{\xi\eta\eta}\eta_y & x_{\xi\eta\eta}\xi_y+x_{\eta\eta\eta}\eta_y \\ y_{\xi\xi\xi}\xi_y+y_{\xi\xi\eta}\eta_y  & y_{\xi\eta\eta}\xi_y+y_{\eta\eta\eta}\eta_y \end{bmatrix}
\end{align}
\end{subequations}
The product rule implies that 
\begin{align*}
|J|_{xx}	&= \xi_{xx}\widetilde{J}_1+\xi_x(\partial_x\widetilde{J}_1)+\eta_{xx}\widetilde{J}_2+\eta_x(\partial_x\widetilde{J}_2) \\ 
|J|_{yy}	&= \xi_{yy}\widetilde{J}_1+\xi_y(\partial_y\widetilde{J}_1)+\eta_{yy}\widetilde{J}_2+\eta_y(\partial_y\widetilde{J}_2)
\end{align*}
The real challenge is differentiating $\widetilde{J}_1$ and $\widetilde{J}_2$. This requires that we must introduce 16 auxiliary functions in total:
\begin{align*}
k_{1x}	&= (x_{\xi\xi\xi}\xi_x+x_{\xi\xi\eta}\eta_x)y_{\eta}+x_{\xi\xi}(y_{\xi\eta}\xi_x+y_{\eta\eta}\eta_x) \\ 
k_{2x}	&= (x_{\xi\xi}\xi_x+x_{\xi\eta}\eta_x)y_{\xi\eta}+x_{\xi}(y_{\xi\xi\eta}\xi_x+y_{\xi\eta\eta}\eta_x) \\
k_{3x}	&= (x_{\xi\xi\eta}\xi_x+x_{\xi\eta\eta}\eta_x)y_{\xi}+x_{\xi\eta}(y_{\xi\xi}\xi_x+y_{\xi\eta}\eta_x) \\
k_{4x}	&= (x_{\xi\eta}\eta_x+x_{\eta\eta}\eta_x)y_{\xi\xi}+x_{\eta}(y_{\xi\xi\xi}\xi_x+y_{\xi\xi\eta}\eta_x) \\
k_{5x}	&= (x_{\xi\xi\eta}\xi_x+x_{\xi\eta\eta}\eta_x)y_{\eta}+x_{\xi\eta}(y_{\xi\eta}\xi_x+y_{\eta\eta}\eta_x) \\
k_{6x}	&= (x_{\xi\xi}\xi_x+x_{\xi\eta}\eta_x)y_{\eta\eta}+x_{\xi}(y_{\xi\eta\eta}\xi_x+y_{\eta\eta\eta}\eta_x) \\
k_{7x}	&= (x_{\xi\eta\eta}\xi_x+x_{\eta\eta\eta}\eta_x)y_{\xi}+x_{\eta\eta}(y_{\xi\xi}\xi_x+y_{\xi\eta}\eta_x) \\
k_{8x}	&= (x_{\xi\eta}\eta_x+x_{\eta\eta}\eta_x)y_{\xi\eta}+x_{\eta}(y_{\xi\xi\eta}\xi_x+y_{\xi\eta\eta}\eta_x) \\
k_{1y}	&= (x_{\xi\xi\xi}\xi_y+x_{\xi\xi\eta}\eta_y)y_{\eta}+x_{\xi\xi}(y_{\xi\eta}\xi_y+y_{\eta\eta}\eta_y) \\
k_{2y}	&= (x_{\xi\xi}\xi_y+x_{\xi\eta}\eta_y)y_{\xi\eta}+x_{\xi}(y_{\xi\xi\eta}\xi_y+y_{\xi\eta\eta}\eta_y) \\
k_{3y}	&= (x_{\xi\xi\eta}\xi_y+x_{\xi\eta\eta}\eta_y)y_{\xi}+x_{\xi\eta}(y_{\xi\xi}\xi_y+y_{\xi\eta}\eta_y) \\
k_{4y}	&= (x_{\xi\eta}\eta_y+x_{\eta\eta}\eta_y)y_{\xi\xi}+x_{\eta}(y_{\xi\xi\xi}\xi_y+y_{\xi\xi\eta}\eta_y) \\
k_{5y}	&= (x_{\xi\xi\eta}\xi_y+x_{\xi\eta\eta}\eta_y)y_{\eta}+x_{\xi\eta}(y_{\xi\eta}\xi_y+y_{\eta\eta}\eta_y) \\
k_{6y}	&= (x_{\xi\xi}\xi_y+x_{\xi\eta}\eta_y)y_{\eta\eta}+x_{\xi}(y_{\xi\eta\eta}\xi_y+y_{\eta\eta\eta}\eta_y) \\
k_{7y}	&= (x_{\xi\eta\eta}\xi_y+x_{\eta\eta\eta}\eta_y)y_{\xi}+x_{\eta\eta}(y_{\xi\xi}\xi_y+y_{\xi\eta}\eta_y) \\
k_{8y}	&= (x_{\xi\eta}\eta_y+x_{\eta\eta}\eta_y)y_{\xi\eta}+x_{\eta}(y_{\xi\xi\eta}\xi_y+y_{\xi\eta\eta}\eta_y) 
\end{align*}
\clearpage\noindent
Thus, we can define the four essential derivatives as
\begin{subequations}
\begin{align}
\partial_x\widetilde{J}_1	&= k_{1x}+k_{2x}-k_{3x}-k_{4x} \\
\partial_y\widetilde{J}_1	&= k_{1y}+k_{2y}-k_{3y}-k_{4y} \\
\partial_x\widetilde{J}_2	&= k_{5x}+k_{6x}-k_{7x}-k_{8x} \\
\partial_y\widetilde{J}_2	&= k_{5y}+k_{6y}-k_{7y}-k_{8y}
\end{align}
\end{subequations}
A similar easier derivation is valid for $U$:
\begin{align*}
U_{xx}	&= \begin{bmatrix} (U_{\xi\xi}\xi_x+U_{\xi\eta}\eta_x)\xi_x+U_{\xi}\xi_{xx}+(U_{\xi\eta}\xi_x+U_{\eta\eta}\eta_x)\eta_x+U_{\eta}\eta_{xx} \\ (V_{\xi\xi}\xi_x+V_{\xi\eta}\eta_x)\xi_x+V_{\xi}\xi_{xx}+(V_{\xi\eta}\xi_x+V_{\eta\eta}\eta_x)\eta_x+V_{\eta}\eta_{xx} \end{bmatrix} \\
		&= \begin{bmatrix} U_{\xi\xi}\xi_x+U_{\xi\eta}\eta_x & U_{\xi\eta}\xi_x+U_{\eta\eta}\eta_x \\ V_{\xi\xi}\xi_x+V_{\xi\eta}\eta_x & V_{\xi\eta}\xi_x+V_{\eta\eta}\eta_x \end{bmatrix}\begin{bmatrix}\xi_x \\ \eta_x \end{bmatrix} + \begin{bmatrix} U_{\xi} & U_{\eta} \\ V_{\xi} & V_{\eta} \end{bmatrix}\begin{bmatrix}\xi_{xx} \\ \eta_{xx} \end{bmatrix}
\end{align*}
Hence, the double-derivatives of $U$ and $V$ are given by
\begin{subequations}
\begin{align}
\begin{bmatrix} U \\ V \end{bmatrix}_{xx}	&= \begin{bmatrix} U_{\xi\xi}\xi_x+U_{\xi\eta}\eta_x & U_{\xi\eta}\xi_x+U_{\eta\eta}\eta_x \\ V_{\xi\xi}\xi_x+V_{\xi\eta}\eta_x & V_{\xi\eta}\xi_x+V_{\eta\eta}\eta_x \end{bmatrix}\begin{bmatrix}\xi_x \\ \eta_x \end{bmatrix} + \begin{bmatrix} U_{\xi} & U_{\eta} \\ V_{\xi} & V_{\eta} \end{bmatrix}\begin{bmatrix}\xi_{xx} \\ \eta_{xx} \end{bmatrix} \\
\begin{bmatrix} U \\ V \end{bmatrix}_{yy}	&= \begin{bmatrix} U_{\xi\xi}\xi_y+U_{\xi\eta}\eta_y & U_{\xi\eta}\xi_y+U_{\eta\eta}\eta_y \\ V_{\xi\xi}\xi_y+V_{\xi\eta}\eta_y & V_{\xi\eta}\xi_y+V_{\eta\eta}\eta_y \end{bmatrix}\begin{bmatrix}\xi_y \\ \eta_y \end{bmatrix} + \begin{bmatrix} U_{\xi} & U_{\eta} \\ V_{\xi} & V_{\eta} \end{bmatrix}\begin{bmatrix}\xi_{yy} \\ \eta_{yy} \end{bmatrix}
\end{align}
\end{subequations}
\end{document}